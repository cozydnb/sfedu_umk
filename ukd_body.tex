\begin{spacing}{1.5}
\begin{center}
\ulinepad{\theects} зач. ед., ак. ч. всего: \ulinepad{\arabic{wholeHours}}, в т. ч.: \ulinepad{\arabic{lectureHours}~ч.} лекций, \ulinepad{\arabic{seminarHours}~ч.} практич., \ulinepad{\arabic{labHours}~ч.} лаб.
\end{center} 

\noindent Преподаватель: \fillanswer{\thefullname}

\noindent Кафедра: \fillanswer{\thebychair}

\noindent Курс: \ulinepad{\theyearofstudy} Семестр: \ulinepad{\theglobalterm}

\noindent Направление подготовки: \fillanswer{\theshortdirection}
\end{spacing}
% -- или
%\noindent Направление подготовки: \fillanswer{фундаментальная информатика}

%\noindent \hphantom{Направление подготовки:}\fillanswer{и информационные технологии}

% для математики можно убрать абзац после курса/семестра, то есть сделать их
% в одну строку с направлением

%   ниже использованы две таблицы для описания УКД, это отличается от формата,
%   данного в образце УКД из П248-ОД, однако их формат:
%		1) неразумно использует пространство страницы (широкие 
%			колонки с числами)
%		2) сложно или невозможно воспроизвести одной таблицей в LaTeX
{\small\tabulinesep = 2mm
\begin{longtabu} to \textwidth {|c|X[3,p]|X[c]|X[c]|}%c|
	\hline
	№ 
		&
		\centering \textbf{Виды контрольных мероприятий}
		&
		%\mc{
		\textbf{Текущий контроль}%}
		&
		\textbf{Рубежный контроль}
	\\\hline
		&
		\centering Модуль 1. \getvalue{UnitNames/1}
		&
		20
		&
		40
	\\\hline
	1.
		& 
		Эссе (по теме)
		&
		5
		&
	\\\hline
	2.
		& 
		Контрольная работа
		&
		10
		&
	\\\hline
	3.
		& 
		Тест
		&
		5
		&
	\\\hline
	4.
		& 
		Проектное задание
		&
		&
		10
	\\\hline
		&
		\centering Модуль 2. \getvalue{UnitNames/2}
		&
		20
		&
		40
	\\\hline
	1.
		& 
		???
		&
		???
		&
		???
	\\\hline
		&
		Всего
		&
		40
		&
		20
	\\\hline
\end{longtabu}
\vspace{-\baselineskip}
\begin{longtabu}{|c|X[c]|c|X[2,p]|}
	\hline
	\hphantom{№}
		&
		Бонусные баллы
		&
		до 10
		&
		Бонусные баллыБонусные баллыБонусные баллыБонусные баллыБонусные баллыБонусные баллыБонусные баллыБонусные баллыБонусные баллы
	\\ \hline
		&
		Промежуточная аттестация \textit{в форме экзамена}
		&
		40 баллов
		&
		Указываются виды и формы проведения экзамена и порядок начисления баллов при проведении экзаменационной процедуры
	\\ \hline
\end{longtabu}
}
