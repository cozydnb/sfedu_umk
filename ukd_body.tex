% сверху можно ничего не редактировать
\begin{spacing}{1.5}
\begin{center}
\ulinepad{\theects} зач. ед., ак. ч. всего: \ulinepad{\arabic{wholeHours}}, в т. ч.:
\IfEq{\arabic{lectureHours}}{0}{}{\ulinepad{\arabic{lectureHours}~ч.} лекций,}
\IfEq{\arabic{seminarHours}}{0}{}{\ulinepad{\arabic{seminarHours}~ч.} практич.,}
\IfEq{\arabic{labHours}}{0}{}{\ulinepad{\arabic{labHours}~ч.} лаб.,}
\IfEq{\arabic{distantHours}}{0}{}{\ulinepad{\arabic{distantHours}~ч.} СР}
\end{center}

\noindent Преподаватели: \fillanswer{Юлия Вячеславовна Белякова}

\noindent \hphantom{Преподаватели}\fillanswer{Виталий Николаевич Брагилевский}

\noindent \hphantom{Преподаватели}\fillanswer{\thefullname}

\noindent Кафедра: \fillanswer{\thebychair}

\noindent Курс: \ulinepad{\theyearofstudy} Семестр: \ulinepad{\theglobalterm}

% здесь могут быть сложности с форматированием в зависимости от названия
%	направления
%\noindent Направление подготовки: \fillanswer{прикладная математика и %информатика}
% -- или
\noindent Направление подготовки: \fillanswer{фундаментальная информатика}

\noindent \hphantom{Направление подготовки:}\fillanswer{и информационные технологии}
%
% для математики можно убрать абзац после курса/семестра, то есть сделать их
% в одну строку с направлением
\end{spacing}

%   ниже использованы две таблицы для описания УКД, это отличается от формата,
%   данного в образце УКД из П248-ОД, однако их формат:
%		1) неразумно использует пространство страницы (широкие
%			колонки с числами)
%		2) сложно или невозможно воспроизвести одной таблицей в LaTeX

% внесите перечень мероприятий в рамках каждого модуля и баллы за них
{\small\tabulinesep = 2mm
\begin{longtabu} to \textwidth {|c|X[3,p]|X[c]|X[c]|}%c|
	\hline
	№
		&
		\centering \textbf{Виды контрольных мероприятий}
		&
		%\mc{
		\textbf{Текущий контроль}%}
		&
		\textbf{Рубежный контроль}
	\\\hline
		&
		\centering Модуль 1. \getvalue{UnitNames/1}
		&
		32
		&
		40
	\\\hline
	1.
		&
		Лабораторные работы
		&
		15
		&
	\\\hline
	2.
		&
		Домашние задания по материалам лабораторных
		&
		5
		&
	\\\hline
	3.
		&
		Домашние задания по материалам лекций
		&
		12
		&
	\\\hline
	4.
		&
		Контрольная работа по материалам лекций
		&
		&
		20
	\\\hline
	5.
		&
		Контрольная лабораторная работа
		&
		&
		20
	\\\hline
		&
		\centering Модуль 2. \getvalue{UnitNames/2}
		&
		13
		&
		15
	\\\hline
	1.
		&
		Лабораторные работы
		&
		4
		&
	\\\hline
	2.
		&
		Домашние задания по материалам лабораторных
		&
		2
		&
	\\\hline
	3.
		&
		Домашние задания по материалам лекций
		&
		7
		&
	\\\hline
	4.
		&
		Контрольная работа по материалам лекций
		&
		&
		15
	\\\hline
		&
		Всего
		&
		45
		&
		55
	\\\hline
\end{longtabu}
}
