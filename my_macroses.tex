% Бланки
\newcommand{\uline}[1]{\underline{\makebox[#1]{}}}
\newcommand{\ugap}{\uline{.5cm}}
\newcommand{\ulinepad}[1]{%
\underline{\makebox[.3cm]{\vphantom{#1}}}%
\underline{#1}%
\underline{\makebox[.3cm]{\vphantom{#1}}}%
}

\newcommand{\fillanswer}[1]{%
  \unskip\ \lhrulefill%
  \makebox[0pt]{#1}%
  \lhrulefill\ \ignorespaces%
}

\newcommand{\lhrulefill}{%
  \leavevmode
  \leaders\hrule height -.3ex depth \dimexpr .3ex+.4pt\relax % define the leader
  \hskip\glueexpr\fill/2\relax\relax % how much it should extend
  \kern0pt
}

% ----------------------------------------------------------------
% Настройка переносов и разрывов страниц

\binoppenalty = 10000      % Запрет переносов строк в формулах
\relpenalty = 10000        %

\sloppy                    % Не выходить за границы бокса
%\tolerance = 400          % или более точно
\clubpenalty = 10000       % Запрет разрывов страниц после первой
\widowpenalty = 10000      % и перед предпоследней строкой абзаца

% ----------------------------------------------------------------

% интерфейс для сохранения значений, которые будут использоваться в 
%	нескольких местах
\usepackage{pgfkeys}
\newcommand{\setvalue}[1]{\pgfkeys{/variables/#1}}
\newcommand{\getvalue}[1]{\pgfkeysvalueof{/variables/#1}}
\newcommand{\declare}[1]{%
 \pgfkeys{
  /variables/#1.is family,
  /variables/#1.unknown/.style = {\pgfkeyscurrentpath/\pgfkeyscurrentname/.initial = ##1}
 }%
}

% Средства для сохранения имён модулей
% пространство имён переменных для хранения имён модулей 
% (должны быть заданы в unit_names.tex)
\declare{UnitNames/}

\newcommand{\myunitname}[1]{%
	\stepcounter{myunitCounter}
	\setvalue{UnitNames/\themyunit = #1} % сохранили название
}
% Счётчик модулей
\newcounter{myunitCounter}
\newcommand{\themyunit}{\arabic{myunitCounter}}

% Макросы для задания общей инфомрации в файле general_info.tex
\renewcommand \date [1] %
{ %
  \newcommand \thedate {#1} %
} %

\newcommand \course [1] %
{ %
  \newcommand \thecourse {<<#1>>} %
} %

\newcommand \direction [1] %
{ %
  \newcommand \thedirection {#1} %
} %

\newcommand \degree [1] %
{ %
  \newcommand \thedegree {#1} %
} %

\newcommand \bychair [1] %
{ %
  \newcommand \thebychair {#1} %
} %

\newcommand \yearofstudy [1] %
{ %
  \newcommand \theyearofstudy {#1} %
} %

\newcommand \term [1] %
{ %
  \newcommand \theterm {#1} %
} %

\newcommand \globalterm [1] %
{ %
  \newcommand \theglobalterm {#1} %
} %

\newcommand \form [1] %
{ %
  \newcommand \theform {#1} %
} %

\newcommand \createdby [1] %
{ %
  \newcommand \thecreatedby {#1} %
} %

\newcommand \fullname [1] %
{ %
  \newcommand \thefullname {#1} %
} %

\newcommand \ects [1] %
{ %
  \newcommand \theects {#1} %
} %

\newcommand \contol [1] %
{ %
  \newcommand \thecontol {#1} %
} %

\newcommand \shortname [1] %
{ %
  \newcommand \theshortname {#1} %
} %

\newcommand \shortdirection [1] %
{ %
  \newcommand \theshortdirection {#1} %
} %

% Команды для задания расчасовки модулей (myunithours)

\usepackage{calc}
\usepackage{array}
\usepackage{tabu}

% величины, вычисляемые относительно заданных часов лек./практ./лаб./СР
\newcounter{wholeUnitHours}  % общее число часов

% счётчик модулей
\newcounter{myunitIdx}
\setcounter{myunitIdx}{0}
 
% величины для последней строки таблицы (Итого)
\newcounter{wholeHours}
\newcounter{lectureHours}
\newcounter{seminarHours}
\newcounter{labHours}
\newcounter{distantHours}
\newcounter{hallHours}

\setcounter{wholeHours}{0}
\setcounter{lectureHours}{0}
\setcounter{seminarHours}{0}
\setcounter{labHours}{0}
\setcounter{distantHours}{0}
\setcounter{hallHours}{0}

% если получилось 0 часов, то не пишем (для последней строки -- Итого)
\newcommand{\ifZeroErase}[1]{%
	\IfEq{#1}{0}{}{#1}
}

% будем запоминать значения часов по каждому модулю, потому что
% вначале нужно сгенерировать таблицу с суммами, а потом -- с часами

% пространства имён для часов по каждому модулю
\declare{UnitLectureHours/}
\declare{UnitSeminarHours/}
\declare{UnitLabHours/}
\declare{UnitDistantHours/}

\newcommand{\myunithours}[4]{%
	% #1 -- лекц, #2 -- практ., #3 -- лаб., #4 -- СР
	\setcounter{wholeUnitHours}{#1 + #2 + #3 + #4}%
%
	% пересчитываем общие суммы
	\addtocounter{wholeHours}{\value{wholeUnitHours}}%
	\addtocounter{lectureHours}{#1 + 0}%
	\addtocounter{seminarHours}{#2 + 0}%
	\addtocounter{labHours}{#3 + 0}%
	\addtocounter{distantHours}{#4}%
	\addtocounter{hallHours}{#1 + #2 + #3}
%
	% запоминаем значения часов для текущего модуля
	\stepcounter{myunitIdx}
	\setvalue{UnitLectureHours/\arabic{myunitIdx} = #1}
	\setvalue{UnitSeminarHours/\arabic{myunitIdx} = #2}
	\setvalue{UnitLabHours/\arabic{myunitIdx} = #3}
	\setvalue{UnitDistantHours/\arabic{myunitIdx} = #4}
}

