\documentclass[fontsize=14pt]{scrartcl} %

\usepackage {cmap}
\usepackage [T2A] {fontenc}
\usepackage [utf8] {inputenc}
\usepackage [english, russian] {babel}
\usepackage{pbox}      	% боксы на титульных страницах
\usepackage{lipsum}	   	% Рыба
\usepackage{url}		% Гиперссылки
\usepackage{pdflscape}
\usepackage{graphicx}

% Таблицы
\usepackage[singlelinecheck=off,font=bf,labelfont=bf,labelsep=period]{caption}
\captionsetup[table]{name=}
\renewcommand\thetable{\arabic{section}.\arabic{table}}

% Отступы
\usepackage{setspace}      % Межстрочный интервал
\onehalfspacing
\usepackage{indentfirst}
\frenchspacing
\parindent=1.25cm


% Поля
\usepackage %
[ %
  left = 2.5cm,
  right = 1.5cm,
  top = 2cm,
  bottom = 1.27cm,
  includefoot,
  footskip = 1cm
] %
  {geometry}

% Шрифт
%\usepackage{paratype}
%\usepackage{libertine}
\usepackage{droid}        % имеется капитель и даже полужирная капитель
\usepackage{textcase}    % Преобразование к верхнему регистру

\input my_macroses.tex

% Вид заголовков
\renewcommand{\thesection}{\Roman{section}}    % римские номера для разделов

% вид заголовка раздела:
% \KOMAoption{headings}{normal} % уменьшить кегль
\addtokomafont{section}{\scshape\centering}

% вид заголовка подраздела:
%% серия грязных хаков. Идея в реализации подраздела через параграфы.
\setcounter{secnumdepth}{5}
% \setcounter{tocdepth}{5} % если нужно видеть подразделы в Содержании
\newcounter{ssectCounter}[section]
\renewcommand{\theparagraph}{\arabic{section}.\arabic{ssectCounter}}
\usepackage{etoolbox}
% команда для подраздела: \ssect или \ssect[Заголовок]
\newcommand{\ssect}[1][]{%
	\stepcounter{ssectCounter}
	\ifstrempty{#1}{%
		\setkomafont{paragraph}{\normalfont}
		\paragraph{}
	}{%
		\setkomafont{paragraph}{\normalfont\bfseries}
		\paragraph{#1}
	}%
}
% теперь параграф это практически подраздел и с точки зрения toc
\makeatletter
\renewcommand*\l@paragraph{\bprot@dottedtocline{2}{1.8em}{2.2em}}
\makeatother

\newcommand {\vgap} {\vspace{0.7\baselineskip}} %

% титульная страница РПД
\AtBeginDocument %
{ %
  \thispagestyle {empty}
  %
  \begin{spacing}{1.1}
  \begin {center}
    %
    {\small
    \textsc{Минобрнауки России}

    \vgap
    Федеральное государственное автономное образовательное
    учреждение высшего профессионального образования

    \textsc{Южный федеральный университет}

    \vgap
    Институт математики, механики и компьютерных наук им.~И.\,И.~Воровича}
    %
  \end {center}
  %
    \vgap
	\hfill
	\begin{minipage}{0.45\textwidth}\small
	\hfill УТВЕРЖДАЮ

	\vspace{.3cm}
	Руководитель направления

	\vspace{1cm}
        \hspace{2.7cm}/

	\hrule

	\vspace{.2cm}
        \hspace{.3cm} (\emph{подпись}) \hfill (\emph{Ф.И.О.}) \hspace{1.5cm}

	\vspace{.5\baselineskip}
	<<\uline{\widthof{XXXX}}>> \uline{\widthof{XXXXXXXXXXXXXXXX}}
         20\ugap\,г.
	\end{minipage}

  \vgap\vgap
  \begin {center}
    {\large\bfseries
    \textsc{Рабочая программа дисциплины}

    \thecourse\par
    }

    \vgap\vgap
    Направление подготовки\\
    \thedirection

    \vgap
    Уровень образования\\
    \ulinepad{\thedegree}

    \vgap
    Форма обучения \ulinepad{\theform}

    \vgap\vgap
    {\small
    \begin{tabularx}{\textwidth}{p{0.4\textwidth}X}
        \centering Программа разработана
                & \uline{5cm}\hfill \theshortname{} \thedignity{}\\
        \\
                & \multicolumn{1}{>{\raggedright}X}{\raggedright
                Рекомендована к утверждению на заседании кафедры
                \thebychair{}}
                \\
        \\
                & Протокол заседания №\uline{.6cm} от \uline{3cm}\\[1cm]
                &
                \begin{tabular}{@{}lcc}
                Зав. кафедрой   & \hspace{2.5cm}   & \hfill В.\,С.~Пилиди\\\cline{2-3}
                                &(\emph{подпись})      & (\emph{Ф.И.О.})\\
                \end{tabular}
    \end{tabularx}
    }

  \end {center}
  %
  \vspace {\fill}
  %
  \begin {center}
    %
    Ростов-на-Дону~--- \thedate
    %
  \end {center}
  %
  \end{spacing}
  \clearpage\normalsize
} %

% Настройка списков
\usepackage {enumitem}

% Таблицы
\usepackage{tabularx}
\usepackage{longtable}
%\usepackage{float}
%\restylefloat{longtable}

% Команда для описания модуля (myunit)
% Вначале создаём новый тип перечня на базе description
%   с горизонтальными линиями после пунктов
%   http://tex.stackexchange.com/q/44205/7460 много кода :(

\usepackage{framed}
\usepackage{letltxmacro}
\usepackage{xstring}

\newtoggle{IsFirstItem}  % so we don't add rule above first item
\toggletrue{IsFirstItem} % personal preference: initialize variables
						 %           explicetly at time of defintion

\newcommand*{\WernersHRule}{ %
   \par\kern\dimexpr.7\itemsep-\parskip-.3\baselineskip\relax%
   \hrulefill%\rule{\textwidth}{.4mm}%
   \par\kern\dimexpr.3\itemsep-.3\parskip-.3\baselineskip\relax%
}%

\LetLtxMacro{\OriginalItem}{\item}         % store existing definition of \item
\newcommand{\ItemWithRuleAbove}[1][]{%
    \iftoggle{IsFirstItem}{}{\WernersHRule}% only add \hrule if not first item
    \IfStrEq{#1}{}{\OriginalItem }{\OriginalItem [#1]}%
    \togglefalse{IsFirstItem}%
}%

% наконец, сам новый тип перечня на базе description:
\newenvironment{MyUnitDescription}{%
    \toggletrue{IsFirstItem}%
    \let\item\ItemWithRuleAbove%
    \begin{description}[topsep=0pt]%
}{%
    \end{description}%
}%

% Увеличенная толщина рамки для информации о модуле
\setlength{\FrameRule}{.6mm}

% Сохраняем имена модулей для использования в следующем пункте

% Макрос описания одного модуля
\newcommand{\myunit}[2]{%
	\stepcounter{myunitCounter}
	\IfEq{\themyunit}{1}{\mbox{}
	 % маленький хак, чтобы описание первого модуля не всплыло выше заголовка
	}{}
	\begin{framed}
	\begin{MyUnitDescription}
		\item [Номер модуля:] \themyunit
		\item [Название модуля:] \getvalue{UnitNames/\themyunit}
		\item [Содержание модуля] #1
		\item [Формы текущего контроля] #2
	\end{MyUnitDescription}
	\end{framed}
}

% Команда для печати расчасовки модулей (printhours)

\usepackage{forloop}

\newcommand{\printhours}{
	Общая трудоемкость дисциплины составляет
	\ulinepad{\theects} зач.~ед. (\ulinepad{\arabic{wholeHours}} часов).

	% Первая таблица с суммами
	%	оформление:
	\renewcommand{\arraystretch}{1.5}
	\setlength\LTleft{0pt}
	\setlength\LTright{0pt}

	\tabulinesep = 1mm
	\begin{longtabu} to \textwidth {|X[4,p] | X[1,c,p]|} \hline
	\textsc{Вид работы} & \textsc{%
		\begin{tabular}{cc}
			Трудоемкость\\(часов)
		\end{tabular}
	} 															\\ \hline
	\textbf{Общая трудоемкость} 	& \arabic{wholeHours} 		\\ \hline
	\textbf{Аудиторная работа:}		& \arabic{hallHours}	 	\\ \hline % всегда равна самостоятельной, поэтому отдельно не считали
	\hspace{1.25cm}Лекции (Л) 		& \arabic{lectureHours} 	\\ \hline
	\hspace{1.25cm}Практические занятия (ПЗ)
									& \arabic{seminarHours} 	\\ \hline
	\hspace{1.25cm}Лабораторные работы (ЛР)
									& \arabic{labHours} 		\\ \hline
	\textbf{Самостоятельная работа:}& 							\\ \hline
	\hspace{1.25cm}Самоподготовка (проработка и повторение лекционного материала и материала учебников и учебных пособий, подготовка к лабораторным и практическим занятиям, коллоквиумам, рубежному контролю и т.д.)%
									& \arabic{distantHours} 	\\ \hline
	Вид итогового контроля (зачет, экзамен)%
									& \thecontol				\\ \hline
	\end{longtabu}

	% Вторая таблица с подробной расчасовкой
	\begin{longtabu} to \textwidth {|c|X[p]|c|c|c|c|c|} % .98  это грязный хак: не знаю, почему она выходит шире, чем нужно без него
		\caption*{\textsc{Модули дисциплины, изучаемые в \theglobalterm{} семестре}} \\

		\hline
		&
		& \multicolumn{5}{c|}{\bfseries Количество часов} \\ \cline{3-7}
		%
	\bfseries
	\begin{tabular}{@{}c@{}}
		№ \\ модуля
	\end{tabular}
		& \centering\bfseries Наименование модулей
	    & \bfseries Всего
	    & \multicolumn{3}{c|}{\bfseries%
		    \begin{tabular}{@{}c@{}}
			Аудиторная \\ работа
			\end{tabular}}
	    & \bfseries
	    	\begin{tabular}{@{}c@{}}
			Внеауд. \\ работа \\ СР
			\end{tabular} \\ \cline{4-6}
			%
	&
		&
		& \bfseries Л
		& \bfseries ПЗ
		& \bfseries ЛР
		&
		\\ \hline
	%
	% Цикл генерации строк таблицы с часами для каждого модуля
	\stepcounter{myunitCounter}
	\forloop{myunitIdx}{1}{\value{myunitIdx} < \value{myunitCounter}} {%
		\setcounter{wholeUnitHours}{%
			\getvalue{UnitDistantHours/\arabic{myunitIdx}}
			+ \getvalue{UnitLectureHours/\arabic{myunitIdx}}
			+ \getvalue{UnitSeminarHours/\arabic{myunitIdx}}
			+ \getvalue{UnitLabHours/\arabic{myunitIdx}}}%
		\IfEq{\arabic{myunitIdx}}{1}{\hspace{-1mm}}{} % ужасный хак
		\arabic{myunitIdx}
			& \getvalue{UnitNames/\arabic{myunitIdx}}
			& \arabic{wholeUnitHours}
			& \getvalue{UnitLectureHours/\arabic{myunitIdx}}
			& \getvalue{UnitSeminarHours/\arabic{myunitIdx}}
			& \getvalue{UnitLabHours/\arabic{myunitIdx}}
			& \getvalue{UnitDistantHours/\arabic{myunitIdx}}
			\\ \hline
	}
	&	Итого:
		& \arabic{wholeHours}
		& \ifZeroErase{\arabic{lectureHours}}
		& \ifZeroErase{\arabic{seminarHours}}
		& \ifZeroErase{\arabic{labHours}}
		& \arabic{distantHours}
		\\ \hline
	\end{longtabu}
}

% Печать информации о лабораторных (при необходимости)

\newcounter{mylabCounter}
\setcounter{mylabCounter}{0}

\newcounter{mylabUnitCounter}
\setcounter{mylabUnitCounter}{1}

\newcommand{\incMyunit}[1][1]{%
	\addtocounter{mylabUnitCounter}{#1}
}

\newcommand{\mylab}[2]{%
	\stepcounter{mylabCounter}

	\IfEq{\arabic{mylabCounter}}{1}{}{\hspace{-.1cm}}%
	\arabic{mylabCounter}
	& \arabic{mylabUnitCounter}
	& #1
	& #2
		\\ \hline%
}

\newcommand{\mysem}[2]{\mylab{#1}{#2}}

\newcommand{\printlabs}[1]{%
\mbox{}

\begin{longtabu} to \textwidth {|c|c|X[p]|c|}
	\hline
	\bfseries
	\begin{tabular}{c}
		№ \\ ЛР
	\end{tabular}
		&
		\bfseries
		\begin{tabular}{c}
			№ \\ модуля
		\end{tabular}
		&
		\centering\bfseries Наименование лабораторных работ
		&
		\bfseries
		\begin{tabular}{c}
			Кол-во \\ часов
		\end{tabular}
		\\ \hline
	#1
\end{longtabu}
}

\newcommand{\printseminars}[1]{%
\mbox{}

\setcounter{mylabCounter}{0}
\setcounter{mylabUnitCounter}{1}
\begin{longtabu} to \textwidth {|c|c|X[p]|c|}
	\hline
	\bfseries
	\begin{tabular}{c}
		№ \\ занятия
	\end{tabular}
		&
		\bfseries
		\begin{tabular}{c}
			№ \\ модуля
		\end{tabular}
		&
		\centering\bfseries Тема
		&
		\bfseries
		\begin{tabular}{c}
			Кол-во \\ часов
		\end{tabular}
		\\ \hline
	#1
\end{longtabu}
}

\makeatletter
    \AddEnumerateCounter{\asbuk}{\@asbuk}{м)}
\makeatother

\newcommand{\rusitems}{%
\renewcommand{\labelenumi}{\asbuk{enumi})}
}

\renewcommand{\labelenumii}{\asbuk{enumii})}

% неразрывный дефис; обозначение: "/
\makeatletter
    \defineshorthand[russian]{"/}{\mbox{-}\bbl@allowhyphens}
\makeatother

\endinput
