\documentclass {scrartcl} % [fontsize=14pt] 

\usepackage {cmap}
\usepackage [T2A] {fontenc}
\usepackage [utf8] {inputenc}
\usepackage [english, russian] {babel}
\usepackage{pbox}      % боксы на титульных страницах
\usepackage{lipsum}	   % Рыба

% Отступы
\usepackage{setspace}      % Межстрочный интервал
%\onehalfspacing
\usepackage{indentfirst}
\frenchspacing
\parindent=1.25cm


% Поля
\usepackage %
[ %
  left = 2.5cm,
  right = 1.5cm,
  top = 2cm,
  bottom = 1.27cm,
  includefoot,
  footskip = 1cm
] %
  {geometry}

% Шрифт
%\usepackage{paratype}
%\usepackage{libertine}
\usepackage{droid}        % имеется капитель и даже полужирная капитель
\usepackage{textcase}    % Преобразование к верхнему регистру

% Бланки
\newcommand{\uline}[1]{\underline{\makebox[#1]{}}}
\newcommand{\ugap}{\uline{.5cm}}
\newcommand{\ulinepad}[1]{%
\underline{\makebox[.3cm]{\vphantom{#1}}}%
\underline{#1}%
\underline{\makebox[.3cm]{\vphantom{#1}}}%
}

% Вид заголовков
\renewcommand{\thesection}{\Roman{section}}    % римские номера для разделов

% вид заголовка раздела:
% \KOMAoption{headings}{normal} % уменьшить кегль
\addtokomafont{section}{\scshape\centering} 

% вид заголовка подраздела:
%% серия грязных хаков. Идея в реализации подраздела через параграфы.
\setcounter{secnumdepth}{5} 
% \setcounter{tocdepth}{5} % если нужно видеть подразделы в Содержании
\newcounter{ssectCounter}[section]
\renewcommand{\theparagraph}{\arabic{section}.\arabic{ssectCounter}}
\usepackage{etoolbox}
% команда для подраздела: \ssect или \ssect[Заголовок]
\newcommand{\ssect}[1][]{%
	\stepcounter{ssectCounter}
	\ifstrempty{#1}{%
		\setkomafont{paragraph}{\normalfont} 
		\paragraph{}
	}{%
		\setkomafont{paragraph}{\normalfont\bfseries}
		\paragraph{#1} 
	}%
}
% теперь параграф это практически подраздел и с точки зрения toc
\makeatletter
\renewcommand*\l@paragraph{\bprot@dottedtocline{2}{1.8em}{2.2em}}
\makeatother

% Макросы титульной страницы
\renewcommand \date [1] %
{ %
  \newcommand \thedate {#1} %
} %

\newcommand \course [1] %
{ %
  \newcommand \thecourse {<<#1>>} %
} %

\newcommand \direction [1] %
{ %
  \newcommand \thedirection {#1} %
} %

\newcommand \degree [1] %
{ %
  \newcommand \thedegree {#1} %
} %

\newcommand \bychair [1] %
{ %
  \newcommand \thebychair {#1} %
} %

\newcommand \yearofstudy [1] %
{ %
  \newcommand \theyearofstudy {#1} %
} %

\newcommand \term [1] %
{ %
  \newcommand \theterm {#1} %
} %

\newcommand \form [1] %
{ %
  \newcommand \theform {#1} %
} %

\newcommand \createdby [1] %
{ %
  \newcommand \thecreatedby {#1} %
} %

\newcommand {\vgap} {\vspace{1.2\baselineskip}} %

% титульная страница РПД
\AtBeginDocument %
{ %
  \thispagestyle {empty}\large
  %  
  \begin{spacing}{1.1}
  \begin {center}
    %
    \textsc{Минобрнауки России}

    \vgap
    Федеральное государственное автономное образовательное
    учреждение высшего профессионального образования

    \textsc{Южный федеральный университет}

    \vgap
    Институт математики, механики и компьютерных наук
    %
  \end {center}
  %
    \vgap
	\hfill 
	\begin{minipage}{0.45\textwidth}
	\begin{center}
	Утверждаю

	Директор института

	\vspace{2\baselineskip}
	\hfill М.\,И.~Карякин
	\hrule

	\vspace{.5\baselineskip}
	<<1>> августа 2014\,г. 
	\end{center}
	\end{minipage}
    
  \vgap
  \begin {center}
    {\LARGE\bfseries
    \textsc{Рабочая программа дисциплины}
    
    \thecourse\par
    }

    \vgap
    Направление подготовки\\
    \thedirection

    \vgap
    Профиль подготовки: \uline{5cm}

    \vgap
    Квалификация (степень) выпускника\\
    \ulinepad{\thedegree}

    \vgap
    Кафедра \ulinepad{\thebychair}

    \vgap
    Курс \ulinepad{\theyearofstudy} $\quad$ семестр \ulinepad{\theterm}

    \vgap
    Форма обучения \ulinepad{\theform}

    \vgap
    Программа разработана \hfill
    %
	\begin{minipage}[t]{.65\textwidth}
	\begin{flushleft}
	\thecreatedby
	\end{flushleft}
	\end{minipage}

	\vgap
	\begin{flushleft}
	Рецензент(ы)
	\end{flushleft}
	\uline{.8\textwidth}
  \end {center}
  %
  \vspace {\fill}
  %
  \begin {center}
    %
    Ростов-на-Дону~--- \thedate
    %
  \end {center}
  %
  \end{spacing}
  \clearpage\normalsize
} %

% ----------------------------------------------------------------
% Настройка переносов и разрывов страниц

\binoppenalty = 10000      % Запрет переносов строк в формулах
\relpenalty = 10000        %

\sloppy                    % Не выходить за границы бокса
%\tolerance = 400          % или более точно
\clubpenalty = 10000       % Запрет разрывов страниц после первой
\widowpenalty = 10000      % и перед предпоследней строкой абзаца

% ----------------------------------------------------------------

% Настройка списков
\usepackage {enumitem}
% нумерация кириллическими буквами
\makeatletter
    \AddEnumerateCounter{\asbuk}{\@asbuk}{м)}
\makeatother
\renewcommand{\labelenumi}{\asbuk{enumi})}

% Таблицы
%\usepackage{longtable}
%\usepackage{float}
%\restylefloat{longtable}

% Команда для описания модуля 
% Вначале создаём новый тип перечисления типа description
%   с горизонтальными линиями после функтов
%   http://tex.stackexchange.com/q/44205/7460

\usepackage{framed}
\usepackage{letltxmacro}
\usepackage{xstring}

\newtoggle{IsFirstItem}  % so we don't add rule above first item
\toggletrue{IsFirstItem} % personal preference: initialize variables
						 %           explicetly at time of defintion

\newcommand*{\WernersHRule}{ % 
   \par\kern\dimexpr.7\itemsep-\parskip-.3\baselineskip\relax%
   \hrulefill%\rule{\textwidth}{.4mm}%
   \par\kern\dimexpr.3\itemsep-.3\parskip-.3\baselineskip\relax%
}%


\LetLtxMacro{\OriginalItem}{\item}         % store existing definition of \item
\newcommand{\ItemWithRuleAbove}[1][]{%
    \iftoggle{IsFirstItem}{}{\WernersHRule}% only add \hrule if not first item
    \IfStrEq{#1}{}{\OriginalItem }{\OriginalItem [#1]}%
    \togglefalse{IsFirstItem}%
}%

\newenvironment{MyUnitDescription}{%
    \toggletrue{IsFirstItem}%
    \let\item\ItemWithRuleAbove%
    \begin{description}[topsep=0pt]%
}{%
    \end{description}%
}%

\setlength{\FrameRule}{.6mm}

\newcounter{myunitCounter}
\newcommand{\themyunit}{\arabic{myunitCounter}}

\newcommand{\myunit}[3]{%
\stepcounter{myunitCounter}
\begin{framed}
\begin{MyUnitDescription}
	\item [Номер модуля:] \themyunit
	\item [Название модуля:]	#1
	\item [Содержание модуля] #2
	\item [Формы текущего контроля] #3
\end{MyUnitDescription}
\end{framed} 
}

\endinput