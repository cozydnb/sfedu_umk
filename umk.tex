\documentclass {scrartcl} % [fontsize=14pt] 

\usepackage {cmap}
\usepackage [T2A] {fontenc}
\usepackage [utf8] {inputenc}
\usepackage [english, russian] {babel}

% Отступы
%\usepackage{setspace}      % Межстрочный интервал
%\onehalfspacing
\usepackage{indentfirst}
\frenchspacing
\parindent=1.25cm


% Поля
\usepackage %
[ %
  left = 2.5cm,
  right = 1.5cm,
  top = 2cm,
  bottom = 1.27cm,
  includefoot,
  footskip = 1cm
] %
  {geometry}

% Шрифт
%\usepackage{mathptmx}
%\usepackage{paratype}
%\usepackage{droid}
\usepackage{droid}
%\usepackage{textcase}    % Преобразование к верхнему регистру

% Бланки
\newcommand{\uline}[1]{\rule[0pt]{#1}{0.4pt}}
\newcommand{\ugap}{\uline{1cm}}

% Вид заголовков
\renewcommand{\thesection}{\Roman{section}}    % римские номера для разделов

% вид заголовка раздела:
% \KOMAoption{headings}{normal} % уменьшить кегль
\addtokomafont{section}{\scshape\bfseries\centering} 

% вид заголовка подраздела:
%% серия грязных хаков. Идея в реализации подраздела через параграфы.
\setcounter{secnumdepth}{5}
\setcounter{tocdepth}{5}
\setkomafont{paragraph}{\normalfont} 
\usepackage{etoolbox}
\renewcommand{\theparagraph}{\arabic{section}.\arabic{paragraph}}
\newcommand{\ssect}[1][]{%
	\ifstrempty{#1}{%
		\setkomafont{paragraph}{\normalfont} 
		\paragraph{}
	}{%
		\setkomafont{paragraph}{\normalfont\bfseries}
		\paragraph{#1} 
	}%
}
% теперь параграф это практически подраздел
\makeatletter
\renewcommand*\l@paragraph{\bprot@dottedtocline{2}{1.8em}{2.2em}}
\makeatother


\begin{document}

\tableofcontents

\section{Цели и задачи освоения дисциплины}
Настоящий курс является продолжением курса «Компьютерные науки», который 
читается для студентов-математиков в 1–2 семестре 1 курса. На первом курсе 
студенты осваивают язык Паскаль, изучают его базовые типы, управляющие 
операторы, процедурную модель программирования и знакомятся с методами 
обработки сложных структур данных: массивов, строк, файлов, динамических 
структур. При разработке курса «Компьютерные науки» для 1 семестра 2 курса 
было учтено, что студенты, приступающие к его изучению, уже владеют базовыми 
приемами программирования (на уровне процедурной модели) и знакомы с языком 
Паскаль и современными интегрированными средами для данного языка (Borland 
Delphi, Free Pascal Lazarus, PascalABC.NET).

\section{Место дисциплины в структуре ООП ВПО}

\ssect[fff]

Учебная дисциплина «Компьютерные науки» (2 курс, 1 семестр) относится к профессиональному циклу.

\ssect

Учебная дисциплина «Компьютерные науки» (2 курс, 1 семестр) относится к 
профессиональному циклу.

\ssect[ggg]

Учебная дисциплина «Компьютерные науки» (2 курс, 1 семестр) относится к 
профессиональному циклу.


\end{document}