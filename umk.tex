\input umk_preamble

\input general_info.tex

\begin{document}

%\tableofcontents

\section{Цели и задачи освоения дисциплины}
% цель и задачи!

\section{Место дисциплины в структуре ООП ВПО}

% использовать \ssect[Абв] или \ssect для нумерации подразделов 
% с факультативным заголовком

	\ssect Учебная дисциплина \thecourse{}
(\theyearofstudy~курс, \theterm~семестр) относится к \ulinepad{?????}
% математическому и естественнонаучному / профессиональному
% обычно видно по учебному плану: м. и ес. там обозначен Б2, п. -- Б3 
% учебные планы ЮФУ: http://sfedu.ru/www/view_plans.startup
циклу.

	\ssect % пререквизиты, например:
% Для изучения курса \thecourse (\theyearofstudy~курс, \theterm~семестр)
% студенту достаточно владеть >>>???основами программирования>>>, 
% полученными в
% базовом курсе «Компьютерные науки», 
% который читается в 1–2 семестре 1 курса.

	\ssect % курс является пререквизитом для... например:
% В дальнейшем материал данного курса будет использоваться в ряде курсов,
% изучаемых на 2–4 курсах и предполагающих использование компьютерных
% технологий, в том числе: ???

\section{Требования к результатам освоения содержания дисциплины}

	\ssect
Процесс изучения дисциплины направлен на формирование элементов следующих компетенций в соответствии с ФГОС ВПО (ОС ЮФУ) и ООП ВПО по данному направлению подготовки:
% общий перечень компетенций по направлению подготовки обычно 
% можно найти в госстандарте или в ООП вуза. ФГОСы размещены на одном из 
% специализированных сайтов Минобра. Сейчас это:
% http://fgosvo.ru/fgosvpo/7/6/1/28
% ООП направлений ЮФУ должны быть на офсайте
% на данный момент (середина 2014) это:
% http://sfedu.ru/www/edu.NaprPodg_show?v_snp_date_in=01.01.2009
% существующий на данный момент перечень для ПМИ и ФИИТ вынесен в файл:
% https://docs.google.com/document/d/12WMvvjyVEkF9S5gQVCI26zMnVXGQJcVEv18Qjz_x9yk/edit?usp=sharing
\begin{enumerate}
\rusitems % нумерация кириллическими буквами
	\item общекультурных (ОК):
	\begin{itemize}
		\item ???
	\end{itemize}

	\item профессиональных (ПК):
	\begin{itemize}
		\item ???
	\end{itemize}
\end{enumerate}

В результате освоения дисциплины обучающийся должен:

\textbf{Знать:}
	\begin{itemize}
		\item ???
	\end{itemize}

\textbf{Уметь:}
	\begin{itemize}
		\item ???
	\end{itemize}

\textbf{Владеть:}
	\begin{itemize}
		\item ???
	\end{itemize}

\section{Содержание и структура дисциплины}
	
	\ssect[Содержание модулей дисциплины]

% общая информация о модулях должна быть представлена в данном файле:
\input my_units.tex

% для изложения содержания каждого модуля используется команда myunit,
%   она создаёт бокс с рамкой, это отличается от табличного формата,
%   данного в образце УМК/РПД, однако их формат:
%   1) неразумно использует пространство страницы (пустые колонки)
%   2) сложно или невозможно воспроизвести в LaTeX, 
%		см. http://tex.stackexchange.com/q/192728/7460

%	имена модулей должны быть заданы в файле my_units.tex
%		здесь указывается только содержание и формы контроля

%	Главный корпус требует от 2 до 4 модулей (включительно)


% Модуль 1
\myunit
	{Современные многоуровневые машины. Понятия архитектуры и организации компьютера. Развитие вычислительной техники. Разнообразие компьютеров: от «одноразовых компьютеров» до суперкомпьютеров, закон Мура. Обзор центрального процессора. Логическая эквивалентность аппаратного и программного обеспечения, принцип микропрограммного управления. Проектирование современных процессоров, CISC и RISC. Параллельные вычислительные системы. 
	Оперативная память, вопросы организации ОЗУ: ячейки, порядок байт. Сверхоперативная память: кэш-память. 
	Иерархия памяти. Вторичная память, устройство накопителей на магнитных дисках, геометрия CHS, основные интерфейсы. Помехоустойчивое кодирование. 
	Подсистема ввода-вывода, системные шины, чипсет.}
	{Домашние работы в виде тестов. Контрольная работа в виде теста.}

% Модуль 2
\myunit
	{Цифровой логический уровень. Транзисторы, вентили, логические схемы, печатные платы. Комбинационные схемы, секвенциональные схемы (схемы памяти), тактовые генераторы. Вопросы проектирования системных шин: ширина, перекосы, мультиплексирование, арбитраж, (а)синхронность. Последовательные и параллельные шины.
	Уровень микроархитектуры. Тракт данных, внутренние регистры, организация управляющего устройства в виде микропрограммы. Пример микроархитектуры Mic-1 и реализация простейших инструкций IJVM с её помощью.
	Вопросы проектирования уровня набора команд. Представление числовых типов данных: знаковые целые числа, числа с плавающей точкой (стандарт IEEE~754).}
	{Домашние работы в виде тестов. Контрольная работа в виде теста.}

	\ssect[Структура дисциплины]
% Вступительная фраза и две таблицы с расчасовками:
\printhours

	\ssect[Лабораторные работы]

%\mylab{Имя лабы}{часы}
\printlabs{%
	\mylab{Простейшие программы, арифметика и циклы LOOP}{4}%
	\mylab{Массивы. Условные и безусловные переходы}{4}%
	\mylab{Интерфейс системных вызовов. Простейшие подпрограммы}{4}%
	\mylab{Подпрограммы (продолжение)}{4}%
	\mylab{Контрольная лабораторная работа}{2}%
	%
	\incMyunit
	%
	\mylab{Цепочечные инструкции}{4}%
	\mylab{Работа с файлами}{4}%
	\mylab{Микропрограммирование}{4}%
	\mylab{Контрольная лабораторная работа}{2}%
}

\section{Образовательные технологии}

\section{Оценочные средства для текущего контроля успеваемости и промежуточной аттестации}

\section{Учебно-методическое обеспечение дисциплины}

	\ssect[Основная литература]

\begin{enumerate}
	\item Tanenbaum A., Ostin T. Structured Computer Organization / Prentice Hall; 6 edition (August 4, 2012). 800 p.\\
	Перевод: Таненбаум Э., Остин Т. Архитектура компьютера / 6-е изд.(+CD) — СПб.: Питер, 2013. — 816 с.
\end{enumerate}

	\ssect[Дополнительная литература]
\begin{enumerate}
	\item Stallings W. Computer Organization and Architecture /  Prentice Hall; 9 edition (March 11, 2012). 792 p.\\
	Перевод: Столлингс У. Структурная организация и архитектура компьютерных систем / М: Вильямс, 2002. 896 с.
\end{enumerate}

	\ssect[Список авторских методических разработок]
\begin{enumerate}
	\item А.\,М.~Пеленицын, Н.\,Н.~Ячменёва. Методические указания к практикуму по курсу «Архитектура компьютера» [Электронный ресурс]\\
	\url{http://open-edu.sfedu.ru/node/2622}
\end{enumerate}

	%\ssect[Интернет-ресурсы]

\section{Материально-техническое обеспечение дисциплины}
	
	%\ssect[Учебно-лабораторное оборудование]

	%\ssect[Программные средства]

	%\ssect[Технические и электронные средства]

\section{Учебная карта дисциплины}

\input ukd_body.tex

\end{document}