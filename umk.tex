\input umk_preamble

\date{2014}
\course{Архитектура компьютера}
\direction{Прикладная математика и информатика~--- 01.03.02}
% Фундаментальная информатика и информационные технологии~--- 02.03.02
% Математика~--- 01.03.01
% Механика и математическое моделирование~--- 01.03.03

\degree{бакалавр}
\bychair{информатики и вычислительного эксперимента}
\yearofstudy{2}
\term{2}
\form{очная}
\createdby{А.\,М.~Пеленицыным, ассистентом кафедры информатики и вычислительного эксперимента}

\begin{document}

%\tableofcontents

\section{Цели и задачи освоения дисциплины}
% цель и задачи!

\section{Место дисциплины в структуре ООП ВПО}

% использовать \ssect[Абв] или \ssect для нумерации подразделов 
% с факультативным заголовком

	\ssect Учебная дисциплина \thecourse{}
(\theyearofstudy~курс, \theterm~семестр) относится к \ulinepad{?????} % профессиональный / 
%							 % математическому и естественнонаучному
циклу.

	\ssect % пререквизиты

	\ssect % курс является пререквизитом для...

\section{Требования к результатам освоения содержания дисциплины}

	\ssect
Процесс изучения дисциплины направлен на формирование элементов следующих компетенций в соответствии с ФГОС ВПО (ОС ЮФУ) и ООП ВПО по данному направлению подготовки:

\begin{enumerate}
	\item общекультурных (ОК):
	\begin{itemize}
		\item ???
	\end{itemize}

	\item профессиональных (ПК):
	\begin{itemize}
		\item ???
	\end{itemize}
\end{enumerate}

В результате освоения дисциплины обучающийся должен:

\textbf{Знать:}
	\begin{itemize}
		\item ???
	\end{itemize}

\textbf{Уметь:}
	\begin{itemize}
		\item ???
	\end{itemize}

\textbf{Владеть:}
	\begin{itemize}
		\item ???
	\end{itemize}

\section{Содержание и структура дисциплины}
	
	\ssect[Содержание модулей дисциплины]
% для описания каждого модуля используется команда myunit,
%   она создаёт бокс с рамкой, это отличается от табличного формата,
%   данного в образце УМК/РПД, однако их формат:
%   1) неразумно использует пространство страницы (пустые колонки)
%   2) сложно или невозможно воспроизвести в LaTeX, 
%		см. http://tex.stackexchange.com/q/192728/7460

\myunit{Название модуля Название модуля Название модуля}
	{\lipsum[1-2]}
	{Контроль Контроль Контроль}

\myunit{Название модуля 2 Название модуля 2 Название модуля 2}
	{\lipsum[3-4]}
	{Контроль Контроль Контроль}


	\ssect[Структура дисциплины] Общая трудоемкость дисциплины составляет
	\ulinepad{???} зач.~ед. (\ulinepad{???} часов).

% Задайте рассчасовку каждого модуля по порядку:
% 	\myunithours{лекции}{практика}{лабы},
% 	например: \myunithours{8}{}{6}

\myunithours{8}{}{6}
\myunithours{6}{}{6}

% Две таблицы с расчасовками, в первом аргументе укажите номер семестра,
%   во втором -- вид итогового контроля (зачёт/экзамен)
% Если требуется указать иные формы самостоятельной работы кроме 
%   самоподготовки (например, реферат, эссе, курсовая работа, курсовой проект,
%	расчётно-графическое задание, самостоятельное изучение модулей,
%	контрольная работа), нужно внести их в файл преамбулы (нужное место
%	можно найти поиском по слову Самоподготовка)
\printhours{???}{???}

	%\ssect[Лабораторные работы]

	%\ssect[Практические занятия (семинары)]

\section{Образовательные технологии}

\section{Оценочные средства для текущего контроля успеваемости и промежуточной аттестации}

\section{Учебно-методическое обеспечение дисциплины}

	\ssect[Основная литература]

	%\ssect[Дополнительная литература]

	%\ssect[Список авторских методических разработок]

	%\ssect[Интернет-ресурсы]

\section{Материально-техническое обеспечение дисциплины}
	
	%\ssect[Учебно-лабораторное оборудование]

	%\ssect[Программные средства]

	%\ssect[Технические и электронные средства]

\section{Учебная карта дисциплины}

\end{document}