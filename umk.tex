\input umk_preamble

\input general_info.tex

\begin{document}

%\tableofcontents

\section{Цели и задачи освоения дисциплины}
% цель и задачи!

\section{Место дисциплины в структуре ООП ВПО}

% использовать \ssect[Абв] или \ssect для нумерации подразделов 
% с факультативным заголовком

	\ssect Учебная дисциплина \thecourse{}
(\theyearofstudy~курс, \theterm~семестр) относится к \ulinepad{?????}
% профессиональному /  математическому и естественнонаучному
циклу.

	\ssect % пререквизиты, например:
% Для изучения курса \thecourse (\theyearofstudy~курс, \theterm~семестр)
% студенту достаточно владеть >>>???основами программирования>>>, 
% полученными в
% базовом курсе «Компьютерные науки», 
% который читается в 1–2 семестре 1 курса.

	\ssect % курс является пререквизитом для... например:
% В дальнейшем материал данного курса будет использоваться в ряде курсов,
% изучаемых на 2–4 курсах и предполагающих использование компьютерных
% технологий, в том числе: ???

\section{Требования к результатам освоения содержания дисциплины}

	\ssect
Процесс изучения дисциплины направлен на формирование элементов следующих компетенций в соответствии с ФГОС ВПО (ОС ЮФУ) и ООП ВПО по данному направлению подготовки:
% общий перечень компетенций по направлению подготовки обычно 
% можно найти в госстандарте или в ООП вуза. ФГОСы размещены на одном из 
% специализированных сайтов Минобра. Сейчас это:
% http://fgosvo.ru/fgosvpo/7/6/1/28
% ООП направлений ЮФУ должны быть на офсайте
% на данный момент (середина 2014) это:
% http://sfedu.ru/www/edu.NaprPodg_show?v_snp_date_in=01.01.2009
% существующий на данный момент перечень для ПМИ и ФИИТ вынесен в файл:
% https://docs.google.com/document/d/12WMvvjyVEkF9S5gQVCI26zMnVXGQJcVEv18Qjz_x9yk/edit?usp=sharing
\begin{enumerate}
\rusitems % нумерация кириллическими буквами
	\item общекультурных (ОК):
	\begin{itemize}
		\item ???
	\end{itemize}

	\item профессиональных (ПК):
	\begin{itemize}
		\item ???
	\end{itemize}
\end{enumerate}

В результате освоения дисциплины обучающийся должен:

\textbf{Знать:}
	\begin{itemize}
		\item ???
	\end{itemize}

\textbf{Уметь:}
	\begin{itemize}
		\item ???
	\end{itemize}

\textbf{Владеть:}
	\begin{itemize}
		\item ???
	\end{itemize}

\section{Содержание и структура дисциплины}
	
	\ssect[Содержание модулей дисциплины]

% общая информация о модулях должна быть представлена в данном файле:
\input my_units.tex

% для изложения содержания каждого модуля используется команда myunit,
%   она создаёт бокс с рамкой, это отличается от табличного формата,
%   данного в образце УМК/РПД, однако их формат:
%   1) неразумно использует пространство страницы (пустые колонки)
%   2) сложно или невозможно воспроизвести в LaTeX, 
%		см. http://tex.stackexchange.com/q/192728/7460

%	имена модулей должны быть заданы в файле my_units.tex
%		здесь указывается только содержание и формы контроля

%	Главный корпус требует от 2 до 4 модулей (включительно)


% Модуль 1
\myunit
	{Описание модуля}
	{Контроль Контроль Контроль}

% Модуль 2
\myunit
	{\lipsum[3-4]}
	{Контроль Контроль Контроль}

% Модуль 3 (при необходимости)
\myunit
	{Описание модуля}
	{Контроль Контроль Контроль}

% Модуль 4 (при необходимости)
%\myunit
%	{Описание модуля}
%	{Контроль Контроль Контроль}


	\ssect[Структура дисциплины]
% Вступительная фраза и две таблицы с расчасовками:
\printhours
% Если требуется указать иные формы самостоятельной работы кроме 
%   самоподготовки (например, реферат, эссе, курсовая работа, курсовой проект,
%	расчётно-графическое задание, самостоятельное изучение модулей,
%	контрольная работа), нужно внести их в файл преамбулы (нужное место
%	можно найти поиском по слову Самоподготовка)

	\ssect[Лабораторные работы]

%\mylab{Имя лабы}{часы}
\printlabs{%
	\mylab{Имя 1}{???}%
	\incMyunit % когда надо увеличить номер модуля; 
			   % возможно \incMyunit[N] для увеличения на N
	\mylab{Имя 2}{???2}%
	\incMyunit[2]%
	\mylab{Имя 3}{???3}%
}

	\ssect[Практические занятия (семинары)]

\printseminars{%
	\mysem{Тема 1}{???}%
	\incMyunit[2] % аналогично лабораторным, когда надо увеличить номер модуля
	\mysem{Тема 2}{???2}%
}

\section{Образовательные технологии}

\section{Оценочные средства для текущего контроля успеваемости и промежуточной аттестации}

\section{Учебно-методическое обеспечение дисциплины}

	\ssect[Основная литература]

\begin{enumerate}%\renewcommand{\labelenumi}{\arabic{enumi}.}
	\item ???
\end{enumerate}

	%\ssect[Дополнительная литература]

	%\ssect[Список авторских методических разработок]

	%\ssect[Интернет-ресурсы]

\section{Материально-техническое обеспечение дисциплины}
	
	%\ssect[Учебно-лабораторное оборудование]

	%\ssect[Программные средства]

	%\ssect[Технические и электронные средства]

\section{Учебная карта дисциплины}

\input ukd_body.tex

\end{document}