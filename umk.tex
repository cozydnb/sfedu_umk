\input umk_preamble

\date{2014}

\begin{document}
  \thispagestyle {empty}\large
  %  
  \begin{spacing}{1.1}
  \begin {center}
    %
    \textsc{Минобрнауки России}

    \vgap
    Федеральное государственное автономное образовательное
    учреждение высшего профессионального образования

    \textsc{Южный федеральный университет}

    \vgap
    Институт математики, механики и компьютерных наук
    %
  \end {center}
  %
    \vgap
	\hfill 
	\begin{minipage}{0.45\textwidth}
	\begin{center}
	Утверждаю

	Директор института

	\vspace{2\baselineskip}
	\hfill М.\,И.~Карякин
	\hrule

	\vspace{.5\baselineskip}
	<<1>> августа 2014\,г. 
	\end{center}
	\end{minipage}
    
  \vgap
  \begin {center}
    {\LARGE\bfseries
    \textsc{Рабочая программа дисциплины}
    
    <<Архитектура компьютера>>\par
    }

    \vgap
    Направление подготовки\\
    Прикладная математика и информатика~--- 01.03.02

    \vgap
    Профиль подготовки: \uline{5cm}

    \vgap
    Квалификация (степень) выпускника\\
    \ulinepad{бакалавр}

    \vgap
    Кафедра \ulinepad{информатики и вычислительного эксперимента}

    \vgap
    Курс \ulinepad{2}    семестр \ulinepad{1}

    \vgap
    Форма обучения \ulinepad{очная}

    \vgap
    Программа разработана \hfill
    %
	\begin{minipage}[t]{.65\textwidth}
	\begin{flushleft}
	А.\,М.~Пеленицыным, ассистентом кафедры информатики и вычислительного эксперимента
	\end{flushleft}
	\end{minipage}

	\vgap
	\begin{flushleft}
	Рецензент(ы)
	\end{flushleft}
	\uline{.8\textwidth}
  \end {center}
  %
  \vspace {\fill}
  %
  \begin {center}
    %
    Ростов-на-Дону~--- \thedate
    %
  \end {center}
  %
  \end{spacing}
  \clearpage\normalsize

%\tableofcontents

\section{Цели и задачи освоения дисциплины}
Настоящий курс является продолжением курса «Компьютерные науки», который 
читается для студентов-математиков в 1–2 семестре 1 курса. На первом курсе 
студенты осваивают язык Паскаль, изучают его базовые типы, управляющие 
операторы, процедурную модель программирования и знакомятся с методами 
обработки сложных структур данных: массивов, строк, файлов, динамических 
структур. При разработке курса «Компьютерные науки» для 1 семестра 2 курса 
было учтено, что студенты, приступающие к его изучению, уже владеют базовыми 
приемами программирования (на уровне процедурной модели) и знакомы с языком 
Паскаль и современными интегрированными средами для данного языка (Borland 
Delphi, Free Pascal Lazarus, PascalABC.NET).

\section{Место дисциплины в структуре ООП ВПО}

\ssect[fff]

Учебная дисциплина «Компьютерные науки» (2 курс, 1 семестр) относится к профессиональному циклу.

\ssect

Учебная дисциплина «Компьютерные науки» (2 курс, 1 семестр) относится к 
профессиональному циклу.

\ssect[ggg]

Учебная дисциплина «Компьютерные науки» (2 курс, 1 семестр) относится к 
профессиональному циклу.


\end{document}