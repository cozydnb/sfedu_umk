\input umk_preamble

\input general_info.tex

\begin{document}

%\tableofcontents

\section{Цели и задачи освоения дисциплины}

Целью данного курса является знакомство студентов с основными принципами функционирования вычислительных систем, а также с проблематикой их проектирования. В рамках курса решаются следующие задачи.
\begin{itemize}
	\item Знакомство с основными этапами исторического развития вычислительной техники.
	\item Изложение современного подхода к изучению компьютерных системы.
	\item Выделение среди общего многообразия компьютеров основных типов и различий в принципах их устройства.
	\item Определение роли каждого из основных компонентов вычислительной системы в её организации, а также проблем, связанных с их проектированием.
	\item Изучение основных примитивов цифровой логики и способов их объединения.
	\item Изложение принципа микропрограммного управления на примере конкретной микроархитектуры.
	\item Определение способа представления чисел в памяти машин.
	\item Описание ключевой роли уровня набора инструкций, как интерфейса между аппаратным и программным обеспечением.
	\item Знакомство с базовыми инструкциями из набора x86 с помощью программирования на языке ассемблера.
\end{itemize}

\section{Место дисциплины в структуре ООП ВПО}

% использовать \ssect[Абв] или \ssect для нумерации подразделов 
% с факультативным заголовком

	\ssect Учебная дисциплина \thecourse{}
(\theyearofstudy~курс, \theterm~семестр) относится к \ulinepad{
% математическому и естественнонаучному%
% / 
профессиональному%
% обычно видно по учебному плану: 
% м. и ес. там обозначен Б2, п. -- Б3 
% учебные планы ЮФУ: http://sfedu.ru/www/view_plans.startup
} циклу.

	\ssect % пререквизиты, например:
Для изучения курса \thecourse{} (\theyearofstudy~курс, \theterm~семестр)
студенту достаточно владеть навыками программирования на одном из императивных языков, например, Pascal или C. К особо важным темам базовых курсов по программированию, понимание которых используется в данном курсе, следует отнести следующие:
\begin{itemize}
	\item указатели и прямая работа с памятью,
	\item организация типа данных <<массив>>,
	\item устройство структуры данных <<линейный односвязный список>>.
\end{itemize}

	\ssect
В дальнейшем материал данного курса может использоваться в ряде курсов,
изучаемых на 3--4 курсах, в том числе: компьютерные сети, операционные системы, теория автоматов и формальных языков, параллельное и многопоточное
программирование.

\section{Требования к результатам освоения содержания дисциплины}

	\ssect
Процесс изучения дисциплины направлен на формирование элементов следующих компетенций в соответствии с ФГОС ВПО (ОС ЮФУ) и ООП ВПО по данному направлению подготовки:
% общий перечень компетенций по направлению подготовки обычно 
% можно найти в госстандарте или в ООП вуза. ФГОСы размещены на одном из 
% специализированных сайтов Минобра. Сейчас это:
% http://fgosvo.ru/fgosvpo/7/6/1/28
% ООП направлений ЮФУ должны быть на офсайте
% на данный момент (середина 2014) это:
% http://sfedu.ru/www/edu.NaprPodg_show?v_snp_date_in=01.01.2009
% существующий на данный момент перечень для ПМИ и ФИИТ вынесен в файл:
% https://docs.google.com/document/d/12WMvvjyVEkF9S5gQVCI26zMnVXGQJcVEv18Qjz_x9yk/edit?usp=sharing
\begin{enumerate}
\rusitems % нумерация кириллическими буквами
	\item общекультурных (ОК):
	\begin{itemize}
		\item ;
		\item ;
		\item ;
	\end{itemize}

	\item профессиональных (ПК):
	\begin{itemize}
		\item ;
		\item ;
		\item ;
		\item ;
	\end{itemize}
\end{enumerate}

В результате освоения дисциплины обучающийся должен

\noindent\textbf{знать:}
	\begin{itemize}[topsep=1mm]
		\item основные этапы развития вычислительной техники,
		\item примеры применения компьютеров в современном обществе,
		\item наиболее широкоупотребительные способы классификации компьютеров,
		\item составляющие части вычислительной системы и проблематику их разработки и взаимодействия,
		\item уровни архитектуры современной вычислительной машины, их назначение и взаимодействие,
		\item основные цифровые логические схемы и шины, используемые в компьютерах,
		\item проблематику микропрограммного управления центральным процессором,
		\item задачи, решаемые уровнем набора инструкций,
		\item способы представления целых знаковых чисел в компьютере,
		\item способы представления рациональных чисел в компьютере: числа с фиксированной точкой, числа с плавающей точкой, стандарт IEEE~754,
		\item ядро набора инструкций x86;
	\end{itemize}

\noindent\textbf{уметь:}
	\begin{itemize}[topsep=1mm]
		\item различать вычислительные машины, относящиеся к разным поколениям,
		\item определять тип вычислительной архитектуры в каждой из основных классификаций: фоннеймановская/гарвардская, CISC/RISC, SISD/SIMD/MISD/MIMD,
		\item определять представление целых беззнаковых чисел в машинах с различной организацией ОЗУ: разным размером байт и разным порядком байт в слове,
		\item использовать простейшие помехоустойчивые коды: проверки чётности, (7,4)"/Хэм\-мин\-га,
		\item прогнозировать поведение простейших цифровых логических схем,
		\item составлять фрагменты микропрограммы для простейших микроархитектур,
		\item определять представление целых знаковых чисел и чисел с плавающей точкой по стандарту IEEE~754 в памяти компьютера,
		\item составлять простейшие программы на языке ассемблера для процессора семейства x86;
	\end{itemize}

\noindent\textbf{владеть:}
	\begin{itemize}[topsep=1mm]
		\item терминологией из области архитектуры компьютера,
		\item средствами ассемблирования программ.
	\end{itemize}

\section{Содержание и структура дисциплины}
	
	\ssect[Содержание модулей дисциплины]

% общая информация о модулях должна быть представлена в данном файле:
\input my_units.tex

% для изложения содержания каждого модуля используется команда myunit,
%   она создаёт бокс с рамкой, это отличается от табличного формата,
%   данного в образце УМК/РПД, однако их формат:
%   1) неразумно использует пространство страницы (пустые колонки)
%   2) сложно или невозможно воспроизвести в LaTeX, 
%		см. http://tex.stackexchange.com/q/192728/7460

%	имена модулей должны быть заданы в файле my_units.tex
%		здесь указывается только содержание и формы контроля

%	Главный корпус требует от 2 до 4 модулей (включительно)


% Модуль 1
\myunit
	{Современные многоуровневые машины. Понятия архитектуры и организации компьютера. Развитие вычислительной техники. Разнообразие компьютеров: от «одноразовых компьютеров» до суперкомпьютеров, закон Мура. Обзор центрального процессора. Логическая эквивалентность аппаратного и программного обеспечения, принцип микропрограммного управления. Проектирование современных процессоров, CISC и RISC. Параллельные вычислительные системы. 
	Иерархия памяти. Оперативная память, вопросы организации ОЗУ: ячейки, порядок байт. Сверхоперативная память: кэш-память. 
	Вторичная память, устройство накопителей на магнитных дисках, геометрия CHS, основные интерфейсы. Помехоустойчивое кодирование. 
	Подсистема ввода-вывода, системные шины, чипсет.}
	{Домашние работы в виде тестов. Контрольная работа в виде теста. Контрольная работа по программированию.}

% Модуль 2
\myunit
	{Цифровой логический уровень. Транзисторы, вентили, логические схемы, печатные платы. Комбинационные схемы, секвенциональные схемы (схемы памяти), тактовые генераторы. Вопросы проектирования системных шин: ширина, перекосы, мультиплексирование, арбитраж, (а)синхронность. Последовательные и параллельные шины.
	Уровень микроархитектуры. Тракт данных, внутренние регистры, организация управляющего устройства в виде микропрограммы. Пример микроархитектуры Mic-1 и реализация простейших инструкций IJVM с её помощью.
	Вопросы проектирования уровня набора команд. Представление числовых типов данных: знаковые целые числа, числа с плавающей точкой (стандарт IEEE~754).}
	{Домашние работы в виде тестов. Контрольная работа в виде теста.
	Контрольная работа по программированию.}

	\ssect[Структура дисциплины]
% Вступительная фраза и две таблицы с расчасовками:
\printhours

	\ssect[Лабораторные работы]

%\mylab{Имя лабы}{часы}
\printlabs{%
	\mylab{Простейшие программы, арифметика и циклы LOOP}{4}%
	\mylab{Массивы. Условные и безусловные переходы}{4}%
	\mylab{Интерфейс системных вызовов. Простейшие подпрограммы}{4}%
	\mylab{Подпрограммы (продолжение)}{4}%
	\mylab{Контрольная лабораторная работа}{2}%
	%
	\incMyunit
	%
	\mylab{Цепочечные инструкции}{4}%
	\mylab{Работа с файлами}{4}%
	\mylab{Микропрограммирование}{4}%
	\mylab{Контрольная лабораторная работа}{2}%
}

\section{Образовательные технологии}

Учебный курс состоит из двух учебных модулей. По окончании каждого модуля проводятся контрольные работы в виде электронных тестов для проверки усвоения теоретического материала и в виде задач для решения на компьютере по аналогии с задачами, выданными в рамках лабораторных работ. Лабораторные работы описаны в пособии [1] п.~\ref{author-res}.

При проведении лекций и практических занятий используются следующие образовательные технологии:
\begin{itemize}
	\item мультимедийные лекции;
	\item электронные формы контроля;
	\item самотестирование студентов.
\end{itemize}

\section{Оценочные средства для текущего контроля успеваемости и промежуточной аттестации}

	\ssect[Примеры вопросов для тестов домашнего задания]

\begin{enumerate}
	\item Программа, которая переводит текст на одном языке (исходном) в текст на другом языке (целевом) называется:
	\begin{enumerate}
		\item виртуальная машина,
		\item интерпретатор,
		\item компилятор.
	\end{enumerate}

	\item Первой электро-механической вычислительной машиной, которая поддерживала числа с плавающей точкой и использовала двоичное представление чисел, была \uline{2cm}.

	\item Что из перечисленного НЕ является отличием универсальных микроконтроллеров от персональных компьютеров?
	\begin{enumerate}
		\item размещение основных подсистем вычислительной системы на единой микросхеме,
		\item борьба за каждую копейку стоимости,
		\item ориентация на гарвардскую архитектуру,
		\item возможность выполнять программы общего назначения,
		\item необходимость работать на низких напряжениях тока.
	\end{enumerate}

	\item Для микропрограммной реализации умножения требовалось многократно провести операнды через \uline{4cm}.

	\item Память некоторого компьютера имеет размер ячейки («байта») в $3$ бита, размер слова~— в $4$ байта и обратный (little) порядок байтов в слове. Запишите двоичное представление слова этого компьютера, содержащего число $321$.

	\item Верхние три уровня иерархии памяти (регистры, кэш и оперативная память) отличаются от последующих тем, что в них входит:
	\begin{enumerate}
		\item энергозависимая память,
		\item память наибольшего объёма,
		\item самая энергозатратная память,
		\item магнитная память.
	\end{enumerate}

	\item На входных линиях 3-мультиплексора имеется следующий сигнал (нумерация контактов слева—направо): 01001110, на управляющих линиях (младшие биты слева): 001. Каково значение выходного сигнала?

	\item Одно из отличий динамической памяти от статической состоит в том, что она
	\begin{enumerate}
		\item более быстрая,
		\item более надёжная,
		\item более объёмная,
		\item асинхронная,
		\item всё вышеперечисленное.
	\end{enumerate}

	\item Пусть операции на шине DDR3 занимают следующее количество тактов работы:
	\begin{itemize}
		\item ACTIVATE = 3 такта,
		\item WRITE = 1 такт,
		\item PRECHARGE = 2 такта после получения банком команды WRITE.
	\end{itemize}
	Предположим, что необходимо записать информацию в две ячейки памяти, расположенные в двух разных банках памяти. Сколько тактов работы шины потребуется для отсылки команд, осуществляющих эти операции, с учётом конвейеризации шины DDR3? В ответе не следует учитывать ожидание окончания последнего PRECHARGE.

	\item Расположение микрокоманд в памяти отличается от расположения инструкций процессора в скомпилированной программе тем, что первые
	\begin{enumerate}
		\item расположены последовательно, в нужном для исполнения очередной инструкции порядке,
		\item расположены в фиксированной последовательности, а порядок исполнения определяется содержимым микрокоманды и динамическими условиями,
		\item загружаются для исполнения в кэш процесора в нужном порядке,
		\item расположены в фиксированной последовательности, а порядок исполнения определяется значением регистра MAR.
	\end{enumerate}

	\item Каково максимальное значение восьмиразрядного числа, хранящегося в системе со смещением на 127?
\end{enumerate}

	\ssect[Примеры задач для лабораторных работ]

\begin{enumerate}
	\item Создайте программу, которая вычитает из числа 3 число 2.

	\item Вычислите значение многочлена $2x^4-3x^2+x-5$
	в точке $x = 7$ (задано в сегменте данных). 
	Указание: четвёртая степень $x$ должна вычисляться с помощью возведения в квадрат второй.

	\item Напишите программу, которая суммирует все элементы массива, заданного в сегменте данных. Результат остаётся в регистре \texttt{AX}.

	\item Найти первый элемент массива, кратный четырём.

	\item Составьте программу, которая печатает на консоль числа от $1$ до $10$ через пробел, а затем печатает символ перехода на новую строку (\texttt{\textbackslash{}n}).

	\item Создайте функцию печати всех элементов данного массива (в качестве аргументов передаётся адрес массива и его длина в словах).

	\item Создайте процедуру SWAP, которая получает адреса
	двух слов и меняет их местами. Помните о необходимости в связи с конвенцией \textit{cdecl} сохранять старые значения регистров, которые вы собираетесь использовать (кроме
	\texttt{AX/CX/DX}).

	\item Создайте функцию \texttt{DIVIDES\_2\_TO\_N}, которая
	в качестве аргументов принимает два целых числа, \texttt{X} и \texttt{N} , и возвращает $1$,
	если \texttt{X} делится на $2^N$, и $0$ в противном случае.

	\item Напишите программу, которая с помощью цепочечных инструкций создаёт копию массива,
	объявленного в сегменте данных. Память для результата выделяется в
	секции BSS.

	\item Создать процедуру \texttt{FILTER}, которая принимает две строки и
	предикат. Она записывает во вторую строку все символы первой, удовлетворяющие данному предикату.

	\item Напишите программу, которая создаёт бинарный файл \texttt{task-1.dat} с целыми числами от $1$ до $10$.

	\item Дан бинарный файл целых чисел, обнулить в нём максимальный
	элемент.

	\item Реализуйте на микропрограммном уровне инструкцию DEC (предлагаемый
	код: 0x16), которая уменьшает слово на вершине стека на единицу. В качестве целевой микроархитектуры используйте Mic-1. Проверку проводите на симуляторе Mi1MMV.

	\item Реализуйте на микропрограммном уровне инструкцию-префикс REP, которая ставится перед любой другой инструкцией X без аргументов и повторяет инструкцию X количество раз, заданное значением на вершине стека.
\end{enumerate}

\section{Учебно-методическое обеспечение дисциплины}

	\ssect[Основная литература]\label{main-lit}

\begin{enumerate}
	\item Tanenbaum A., Ostin T. Structured Computer Organization / Prentice Hall; 6th edition (August 4, 2012). 800 p.\\
	Перевод: Таненбаум Э., Остин Т. Архитектура компьютера / 6-е изд.(+CD) — СПб.: Питер, 2013. — 816 с.
\end{enumerate}

	\ssect[Дополнительная литература]
\begin{enumerate}
	\item Stallings W. Computer Organization and Architecture /  Prentice Hall; 9th edition (March 11, 2012). 792 p.\\
	Перевод: Столлингс У. Структурная организация и архитектура компьютерных систем / М: Вильямс, 2002. 896 с.
\end{enumerate}

	\ssect[Список авторских методических разработок]
	\label{author-res}
\begin{enumerate}
	\item А.\,М.~Пеленицын, Н.\,Н.~Ячменёва. Методические указания к практикуму по курсу «Архитектура компьютера» [Электронный ресурс]\\
	\url{http://open-edu.sfedu.ru/node/2622}
\end{enumerate}

	\ssect[Интернет-ресурсы]\label{online-res}
\begin{enumerate}
	\item Сопроводительные материалы к учебнику [1] п.~\ref{main-lit}:
	\begin{otherlanguage}{english}
	Structured Computer Organization, 6/E 
	\end{otherlanguage}
	[Электронный ресурс]\\
	\url{http://www.pearsonhighered.com/educator/product/Structured-Computer-Organization-6E/9780132916523.page}

	\item Википедия: Свободная энциклопедия, английский раздел [Электронный ресурс]\\
	\url{http://en.wikipedia.org/wiki/Main_Page}
\end{enumerate}

\section{Материально-техническое обеспечение дисциплины}
	
	\ssect[Учебно-лабораторное оборудование]
Лекции проводятся в мультимедийном классе с презентационным оборудованием (проектором и экраном либо интерактивной доской). Лабораторные занятия проводятся в дисплейных классах с персональными компьютерами по числу, не уступающему числу студентов.

	\ssect[Программные средства]

Компьютеры в дисплейных классах должны быть снабжены операционной системой GNU/Linux, желательно в виде дружественного к пользователю дистрибутива (например, Ubuntu Linux).

На компьютерах в дисплейном классе должны быть распакованы в каталог \texttt{/bin} или \texttt{\textasciitilde/bin} три утилиты ассемблирования (as88/t88/s88) из сопроводительных материалов к учебнику Таненбаума (см. [1] в п.~\ref{online-res}). Для выполнения лабораторной работы по микропрограммированию необходимо наличие каталога Mic1MMV (с содержимым) оттуда же. Для его работы требуется виртуальная машина Java версии не ниже 1.4.

Для редактирования кода рекомендуется иметь установленным специализированный редактор с подсветкой синтаксиса языка ассемблера, например Geany или jEdit (оба находятся в частности в репозиториях Ubuntu Linux).

В отдельных случаях возможна работа на компьютерах под управлением операционной системы семейства Windows, однако здесь требуется дополнительная настройка утилит ассемблирования, описанная в методических указаниях~[1] п.~\ref{author-res}.


	\ssect[Технические и электронные средства]

Учёт активности студентов на курсе и основные материалы размещены в системе Moodle, развёрнутой в сети университета по адресу \url{http://edu.mmcs.sfedu.ru}. В дисплейных классах требуется доступ к этому ресурсу посредством браузера актуальной версии. Для проведения контрольных работ необходимо одновременно с доступом к указанному ресурсу ограничить доступ к другим интернет- и интранет-ресурсам, что на данный момент реализовано в лаборатории ММПТ и лаборатории кафедр алгебры и дискретной математики и информатики и вычислительного эксперимента факультета.

\section{Учебная карта дисциплины}

\input ukd_body.tex

\end{document}